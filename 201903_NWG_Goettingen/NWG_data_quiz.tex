\documentclass[8pt,BCOR10mm,oneside,headsepline]{scrartcl}
%\areaset{17cm}{26cm}
\usepackage[a5paper]{geometry}
\usepackage[cm]{fullpage}
\usepackage[ngerman]{babel}
\usepackage[utf8]{inputenc}
\usepackage{graphicx}
\usepackage{svg}
\usepackage{wasysym}% provides \ocircle and \Box
\usepackage{enumitem}% easy control of topsep and leftmargin for lists
\usepackage{color}% used for background color
\usepackage{forloop}% used for \Qrating and \Qlines
\usepackage{ifthen}% used for \Qitem and \QItem
\usepackage{typearea}
\setlength{\topmargin}{-2.2cm}
\setlength\headheight{45pt}
\usepackage{scrpage2}
\usepackage{fancyhdr}
\pagestyle{fancy}
\fancyhf{}
\renewcommand{\headrulewidth}{0pt}
%%\pagestyle{scrheadings}
%%\ihead{G-Node NWG questionnaire}
\ohead{\pagemark}
\chead{}
\cfoot{}
%\thispagestyle{empty}

%%%%%%%%%%%%%%%%%%%%%%%%%%%%%%%%%%%%%%%%%%%%%%%%%%%%%%%%%%%%
%% Beginning of questionnaire command definitions %%
%%%%%%%%%%%%%%%%%%%%%%%%%%%%%%%%%%%%%%%%%%%%%%%%%%%%%%%%%%%%
%% 2010, 2012 by Sven Hartenstein
%% mail@svenhartenstein.de
%% http://www.svenhartenstein.de

%% \Qq = Questionaire question. Oh, this is just too simple. It helps
%% making it easy to globally change the appearance of questions.
\newcommand{\Qq}[1]{\textbf{#1}}

%% \QO = Circle or box to be ticked. Used both by direct call and by
%% \Qrating and \Qlist.
    \newcommand{\QO}{{\huge{$\Box$}}}% or: $\ocircle$

%% \Qrating = Automatically create a rating scale with NUM steps, like
%% this: 0--0--0--0--0.
\newcounter{qr}
\newcommand{\Qrating}[1]{\QO\forloop{qr}{1}{\value{qr} < #1}{---\QO}}

%% \Qline = Again, this is very simple. It helps setting the line
%% thickness globally. Used both by direct call and by \Qlines.
\newcommand{\Qline}[1]{\noindent\rule{#1}{0.6pt}}

%% \Qlines = Insert NUM lines with width=\linewith. You can change the
%% \vskip value to adjust the spacing.
\newcounter{ql}
\newcommand{\Qlines}[1]{\forloop{ql}{0}{\value{ql}<#1}{\vskip0em\Qline{\linewidth}}}

%% \Qlist = This is an environment very similar to itemize but with
%% \QO in front of each list item. Useful for classical multiple
%% choice. Change leftmargin and topsep accourding to your taste.
\newenvironment{Qlist}{%
\renewcommand{\labelitemi}{\QO}
\begin{itemize}[leftmargin=1.5em,topsep=-.5em]
}{%
\end{itemize}
}

%% \Qtab = A "tabulator simulation". The first argument is the
%% distance from the left margin. The second argument is content which
%% is indented within the current row.
\newlength{\qt}
\newcommand{\Qtab}[2]{
\setlength{\qt}{\linewidth}
\addtolength{\qt}{-#1}
\hfill\parbox[t]{\qt}{\raggedright #2}
}

%% \Qitem = Item with automatic numbering. The first optional argument
%% can be used to create sub-items like 2a, 2b, 2c, ... The item
%% number is increased if the first argument is omitted or equals 'a'.
%% You will have to adjust this if you prefer a different numbering
%% scheme. Adjust topsep and leftmargin as needed.
\newcounter{itemnummer}
\newcommand{\Qitem}[2][]{% #1 optional, #2 notwendig
\ifthenelse{\equal{#1}{}}{\stepcounter{itemnummer}}{}
\ifthenelse{\equal{#1}{a}}{\stepcounter{itemnummer}}{}
\begin{enumerate}[topsep=2pt,leftmargin=2.8em]
\item[\textbf{\arabic{itemnummer}#1.}] #2
\end{enumerate}
}

%% \QItem = Like \Qitem but with alternating background color. This
%% might be error prone as I hard-coded some lengths (-5.25pt and
%% -3pt)! I do not yet understand why I need them.
\definecolor{bgodd}{rgb}{0.8,0.8,0.8}
\definecolor{bgeven}{rgb}{0.9,0.9,0.9}
\newcounter{itemoddeven}
\newlength{\gb}
\newcommand{\QItem}[2][]{% #1 optional, #2 notwendig
\setlength{\gb}{\linewidth}
\addtolength{\gb}{-5.25pt}
\ifthenelse{\equal{\value{itemoddeven}}{0}}{%
\noindent\colorbox{bgeven}{\hskip-3pt\begin{minipage}{\gb}\Qitem[#1]{#2}\end{minipage}}%
\stepcounter{itemoddeven}%
}{%
\noindent\colorbox{bgodd}{\hskip-3pt\begin{minipage}{\gb}\Qitem[#1]{#2}\end{minipage}}%
\setcounter{itemoddeven}{0}%
}
}

%%%%%%%%%%%%%%%%%%%%%%%%%%%%%%%%%%%%%%%%%%%%%%%%%%%%%%%%%%%%
%% End of questionnaire command definitions %%
%%%%%%%%%%%%%%%%%%%%%%%%%%%%%%%%%%%%%%%%%%%%%%%%%%%%%%%%%%%%
\lhead{\includesvg[width=120pt]{G-Node_LogoLong.svg}}
\rhead{\includesvg[width=40pt]{GinLogo_text.svg}}
%%\fancyfoot[L]{\fontsize{14}{16} \fontfamily{pzc} \selectfont Questionnaire \thepage}
%%\fancyfoot[R]{\fontsize{14}{16} \fontfamily{pzc} \selectfont Questionnaire \thepage}



\begin{document}

\mbox{}\vspace{1em}
\begin{center}
\textbf{\huge The G-Node Data Quiz}
\end{center}\vskip1em

\noindent How much do you now about data storage, versioning, management and publication? The German Neuroinformatics Node challenges your knowledge! The correct answers and some goodies for correct answers will be revealed on Friday!

\noindent \textbf{For some questions, there may be multiple correct answers.}

\section*{Questions:}

\Qitem{\Qq{In a CERN study, 97 petabytes of data were checked. How many bytes were found silently corrupted after six months:}} 
{\Qtab{1.2cm}{
\QO{ 0 bytes} \hskip0.2cm
\QO{ 500 bytes} \hskip0.2cm
\QO{ 128 megabytes} \hskip0.2cm
\QO{ 7 gigabytes}  \hskip0.2cm
\QO{ 1 petabyte}
 }}

\Qitem{\Qq{Which of the following software tools help you version your code:} }
{\Qtab{1.2cm}{
\QO{ got} \hskip0.5cm
\QO{ svn} \hskip0.5cm
\QO{ git} \hskip0.5cm
\QO{ Mercurial} \hskip0.5cm
\QO{ perl} \hskip0.5cm
\QO{ sting}  }}

\Qitem{\Qq{What does the abbreviation DOI stand for:} }
{\Qtab{1.2cm}{
\QO{ Digital Open Index} \hskip0.4cm
\QO{ Digital Object Index} \hskip0.4cm \\
\QO{ Digital Object Identifier} \hskip0.4cm
\QO{ Domain Object Interface } }}


\Qitem{ \Qq{Which of the following free services help programmers and scientists with code quality:}}
{\Qtab{1.2cm}{
\QO{ travis} \hskip0.5cm
\QO{ pylint} \hskip0.5cm
\QO{ orangebot} \hskip0.5cm
\QO{ jenkins} \hskip0.5cm
\QO{ devcircle} }}


\Qitem{ \Qq{Which aspects are not included in the FAIR principles:}}
{\Qtab{1.2cm}{
\QO{ Free }\hskip0.5cm
\QO{ Accurate } \hskip0.5cm
\QO{ Interoperable } \hskip0.5cm
\QO{ Reliable }\hskip0.5cm \\
\QO{ Findable }\hskip0.5cm
\QO{ Accessible }\hskip0.5cm
\QO{ Identifiable}\hskip0.5cm
\QO{ Reusable }\hskip0.5cm
}}

\Qitem{\Qq{What does metadata mean:} }
{\Qtab{1.2cm}{
\QO{ Data about data } \hskip0.5cm
\QO{ Beyond an item given } \hskip0.5cm
\QO{ Additional work }
\\
\QO{ Understanding how an experiment was conducted }
}}

\Qitem{ \Qq{Which technologies are supported by the GIN data hosting platform for downloading data:} }
{\Qtab{1.2cm}{
\QO{ https} \hskip0.5cm
\QO{ ftp} \hskip0.5cm
\QO{ git} \hskip0.5cm
\QO{ WebDAV}  \hskip0.5cm
\QO{ telnet}
 }}


~\\

\noindent \text{Do you want to know the correct answers or do you want to know more about data management, programming in science, versioning or metadata? The G-Node team will be happy to discuss it in detail and some open source tools that they can offer at the BCOS booth at any time of your convenience during this conference.}
~\\

\noindent \textsl{ Quiz technicalities: Please tear off and keep the number in the lower right corner. Every correctly answered quiz form will be eligible for a drawing of all handed in forms. The first ten drawn lots will win valuable prizes. The drawing will take place at the BCOS booth on Friday, 22.03.2019, at 10:30. Winners can claim their prizes by presenting their questionnaire number until Saturday 12:00 at the BCOS booth. You can hand in a filled in quiz form to the G-Node team at the booth.}

\end{document}
